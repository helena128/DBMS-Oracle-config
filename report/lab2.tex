\documentclass[12pt, a4paper]{article}
\usepackage[T2A]{fontenc}
\usepackage[utf8]{inputenc}
\usepackage[russian]{babel}
\usepackage[top=10mm, bottom=20mm]{geometry}
\usepackage{float}
\usepackage{tikz}
\usepackage{listings}
\lstset{
	language=bash,
	basicstyle=\ttfamily
}
\newcommand{\sectionbreak}{\clearpage}


\begin{document}
	\include{titlepage}
	Цель работы - сконфигурировать базу данных Oracle на выделенном сервере. В процессе конфигурации БД необходимо пользоваться только интерфейсом командной строки и утилитой SQLPlus; использовать графический установщик нельзя.
	\section{Задание}
	\textbf{Порядок конфигурации БД:}\\
	\begin{enumerate}
		\item Задать значения необходимых для конфигурации переменных окружения.
		\item Задать метод аутентификации администратора (зависит от варианта).
		\item Создать конфигурационные файлы, необходимые для инициализации и запуска экземпляра Oracle.
		\item Запустить экземпляр Oracle.
		\item Создать новую базу данных (параметры конфигурации зависят от варианта).
		\item Создать дополнительные табличные пространства (определяются вариантом).
		\item Сформировать представления словаря данных.
	\end{enumerate}
	
	\textbf{Параметры конфигурации Oracle:}\\
	\begin{itemize}
		\item Имя узла: \texttt{db161}.
		\item Точка монтирования: \texttt{/u01/huk07}.
		\item SID: \texttt{\$surname\$name\$groupnumber}, где \texttt{\$surname, \$name} и \texttt{\$groupnumber} - фамилия, имя студента (транслитом) и номер группы.
		\item Метод аутентификации администратора БД: пароль.
		\item Имя БД: \texttt{wetlaw}.
		\item Размер блока данных: 8192 байт.
		\item Размер SGA: 640 МБ.
		\item Кодировка: UTF-8.
		\item Файлы данных табличного пространства SYSTEM:
		\begin{itemize}
			\item \texttt{\$ORADATA/node01/oluke5.dbf.}
			\item \texttt{\$ORADATA/node01/omozo48.dbf.}
		\end{itemize}
		\item Файлы данных табличного пространства SYSAUX:
		\texttt{\$ORADATA/node02/fug40.dbf.}
		\item Файлы данных табличного пространства USERS:
		\texttt{\$ORADATA/node02/ayetohi361.dbf.}
		\item Файлы данных дополнительных табличных пространств:
		\begin{itemize}
			\item \texttt{GOOD\_WHITE\_BIRD:
			\$ORADATA/node04/goodwhitebird01.dbf,
			\$ORADATA/node01/goodwhitebird02.dbf,
			\$ORADATA/node04/goodwhitebird03.dbf.
			\item BUSY\_YELLOW\_DATA:
			\$ORADATA/node01/busyyellowdata01.dbf,
			\$ORADATA/node04/busyyellowdata02.dbf,
			\$ORADATA/node04/busyyellowdata03.dbf,
			\$ORADATA/node01/busyyellowdata04.dbf.
			\item DRY\_BROWN\_OVEN:
			\$ORADATA/node04/drybrownoven01.dbf.}
		\end{itemize}
	\end{itemize}

	\section{Выполнение}
	\texttt{\textbf{init.sh}}\\
	\begin{lstlisting}
#!/bin/sh
export ORACLE_SID="tcheprasovaelenap3402"
export ORACLE_HOME="/u01/app/oracle/product/11.2.0/dbhome_1"
export PATH=${PATH}:${ORACLE_HOME}/bin
export NLS_LANG="AMERICAN_AMERICA.UTF8"
export ORADATA="/u01/huk07/wetlaw"

if [ ! -e "$ORADATA" ]; then
	mkdir -p "$ORADATA"
	mkdir "$ORADATA/node01"
	mkdir "$ORADATA/node02"
	mkdir "$ORADATA/node03"
	mkdir "$ORADATA/node04"
	mkdir "$ORADATA/flash_recovery_area"
fi

cp "init$ORACLE_SID.ora" "$ORACLE_HOME/dbs/"
sqlplus /nolog @create_db.sql
	\end{lstlisting}
	\vspace{2.7cm}
	\textbf{\texttt{inittcheprasovap3402.ora}}\\
	\begin{lstlisting}
DB_NAME = "wetlaw"
DB_BLOCK_SIZE=8192
SGA_MAX_SIZE=640M
DB_RECOVERY_FILE_DEST="/u01/huk07/wetlaw/flash_recovery_area"
DB_RECOVERY_FILE_DEST_SIZE=200M
CONTROL_FILES=(/u01/huk07/wetlaw/ora_control1.ctl, /u01/huk07/wetlaw/ora_control2.ctl)
REMOTE_LOGIN_PASSWORDFILE=NONE
REMOTE_OS_AUTHENT=TRUE
OS_AUTHENT_PREFIX=NULL
	\end{lstlisting}
	\vspace{2.7cm}
	\texttt{\textbf{create\_db.sql}}\\
	\begin{lstlisting}
CONNECT / AS SYSDBA
CREATE SPFILE FROM PFILE = 'inittcheprasovaelenap3402.ora';
STARTUP NOMOUNT;

CREATE DATABASE wetlaw
	LOGFILE	
		GROUP 1
			('/u01/huk07/wetlaw/ora_control1.log')	
			SIZE 100M,
		GROUP 2 
			('/u01/huk07/wetlaw/ora_control2.log')
			SIZE 100M
	CHARACTER SET AL32UTF8
	NATIONAL CHARACTER SET AL16UTF16
	DATAFILE 
		'/u01/huk07/wetlaw/node01/oluke5.dbf'
			SIZE 150M REUSE 
			AUTOEXTEND ON MAXSIZE UNLIMITED,
		'/u01/huk07/wetlaw/node01/omozo48.dbf'
			SIZE 150M REUSE 
			AUTOEXTEND ON MAXSIZE UNLIMITED
	SYSAUX DATAFILE 
		'/u01/huk07/wetlaw/node02/fug40.dbf'
		SIZE 150M 
		REUSE AUTOEXTEND ON MAXSIZE UNLIMITED
	DEFAULT TABLESPACE users DATAFILE
		'/u01/huk07/wetlaw/node02/ayetohi361.dbf'
		SIZE 100M 
		REUSE AUTOEXTEND ON MAXSIZE UNLIMITED
	UNDO TABLESPACE undotbs1 DATAFILE
		'/u01/huk07/wetlaw/node02/undotbs.dbf'
		SIZE 100M REUSE AUTOEXTEND ON
		MAXSIZE UNLIMITED;

CREATE TABLESPACE GOOD_WHITE_BIRD DATAFILE
	'/u01/huk07/wetlaw/node04/goodwhitebird01.dbf'
	SIZE 100M REUSE AUTOEXTEND ON MAXSIZE UNLIMITED,
	'/u01/huk07/wetlaw/node01/goodwhitebird02.dbf'
	SIZE 100M REUSE AUTOEXTEND ON MAXSIZE UNLIMITED,
	'/u01/huk07/wetlaw/node04/goodwhitebird03.dbf'
	SIZE 100M REUSE AUTOEXTEND ON MAXSIZE UNLIMITED;

CREATE TABLESPACE BUSY_YELLOW_DATA DATAFILE
	'/u01/huk07/wetlaw/node01/busyyellowdata01.dbf'
		SIZE 100M REUSE AUTOEXTEND ON MAXSIZE UNLIMITED,
	'/u01/huk07/wetlaw/node04/busyyellowdata02.dbf' 
		SIZE 100M REUSE AUTOEXTEND ON MAXSIZE UNLIMITED,
	'/u01/huk07/wetlaw/node04/busyyellowdata03.dbf'
		SIZE 100M REUSE AUTOEXTEND ON MAXSIZE UNLIMITED,
	'/u01/huk07/wetlaw/node01/busyyellowdata04.dbf'
		SIZE 100M REUSE AUTOEXTEND ON MAXSIZE UNLIMITED;

CREATE TABLESPACE DRY_BROWN_OVEN DATAFILE
	'/u01/huk07/wetlaw/node04/drybrownoven01.dbf' 
	SIZE 100M REUSE AUTOEXTEND ON MAXSIZE UNLIMITED;

@?/rdbms/admin/catalog.sql
@?/rdbms/admin/catproc.sql
@?/sqlplus/admin/pupbld.sql
\end{lstlisting}
	
	\section{Выводы}
	В ходе выполнения данной лабораторной работы был получен опыт настройки базы данных Oracle.
	
\end{document}